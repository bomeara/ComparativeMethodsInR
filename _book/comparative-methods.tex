\documentclass[]{book}
\usepackage{lmodern}
\usepackage{amssymb,amsmath}
\usepackage{ifxetex,ifluatex}
\usepackage{fixltx2e} % provides \textsubscript
\ifnum 0\ifxetex 1\fi\ifluatex 1\fi=0 % if pdftex
  \usepackage[T1]{fontenc}
  \usepackage[utf8]{inputenc}
\else % if luatex or xelatex
  \ifxetex
    \usepackage{mathspec}
  \else
    \usepackage{fontspec}
  \fi
  \defaultfontfeatures{Ligatures=TeX,Scale=MatchLowercase}
\fi
% use upquote if available, for straight quotes in verbatim environments
\IfFileExists{upquote.sty}{\usepackage{upquote}}{}
% use microtype if available
\IfFileExists{microtype.sty}{%
\usepackage{microtype}
\UseMicrotypeSet[protrusion]{basicmath} % disable protrusion for tt fonts
}{}
\usepackage[margin=1in]{geometry}
\usepackage{hyperref}
\hypersetup{unicode=true,
            pdftitle={Comparative Methods},
            pdfauthor={Brian O'Meara},
            pdfborder={0 0 0},
            breaklinks=true}
\urlstyle{same}  % don't use monospace font for urls
\usepackage{natbib}
\bibliographystyle{apalike}
\usepackage{color}
\usepackage{fancyvrb}
\newcommand{\VerbBar}{|}
\newcommand{\VERB}{\Verb[commandchars=\\\{\}]}
\DefineVerbatimEnvironment{Highlighting}{Verbatim}{commandchars=\\\{\}}
% Add ',fontsize=\small' for more characters per line
\usepackage{framed}
\definecolor{shadecolor}{RGB}{248,248,248}
\newenvironment{Shaded}{\begin{snugshade}}{\end{snugshade}}
\newcommand{\KeywordTok}[1]{\textcolor[rgb]{0.13,0.29,0.53}{\textbf{#1}}}
\newcommand{\DataTypeTok}[1]{\textcolor[rgb]{0.13,0.29,0.53}{#1}}
\newcommand{\DecValTok}[1]{\textcolor[rgb]{0.00,0.00,0.81}{#1}}
\newcommand{\BaseNTok}[1]{\textcolor[rgb]{0.00,0.00,0.81}{#1}}
\newcommand{\FloatTok}[1]{\textcolor[rgb]{0.00,0.00,0.81}{#1}}
\newcommand{\ConstantTok}[1]{\textcolor[rgb]{0.00,0.00,0.00}{#1}}
\newcommand{\CharTok}[1]{\textcolor[rgb]{0.31,0.60,0.02}{#1}}
\newcommand{\SpecialCharTok}[1]{\textcolor[rgb]{0.00,0.00,0.00}{#1}}
\newcommand{\StringTok}[1]{\textcolor[rgb]{0.31,0.60,0.02}{#1}}
\newcommand{\VerbatimStringTok}[1]{\textcolor[rgb]{0.31,0.60,0.02}{#1}}
\newcommand{\SpecialStringTok}[1]{\textcolor[rgb]{0.31,0.60,0.02}{#1}}
\newcommand{\ImportTok}[1]{#1}
\newcommand{\CommentTok}[1]{\textcolor[rgb]{0.56,0.35,0.01}{\textit{#1}}}
\newcommand{\DocumentationTok}[1]{\textcolor[rgb]{0.56,0.35,0.01}{\textbf{\textit{#1}}}}
\newcommand{\AnnotationTok}[1]{\textcolor[rgb]{0.56,0.35,0.01}{\textbf{\textit{#1}}}}
\newcommand{\CommentVarTok}[1]{\textcolor[rgb]{0.56,0.35,0.01}{\textbf{\textit{#1}}}}
\newcommand{\OtherTok}[1]{\textcolor[rgb]{0.56,0.35,0.01}{#1}}
\newcommand{\FunctionTok}[1]{\textcolor[rgb]{0.00,0.00,0.00}{#1}}
\newcommand{\VariableTok}[1]{\textcolor[rgb]{0.00,0.00,0.00}{#1}}
\newcommand{\ControlFlowTok}[1]{\textcolor[rgb]{0.13,0.29,0.53}{\textbf{#1}}}
\newcommand{\OperatorTok}[1]{\textcolor[rgb]{0.81,0.36,0.00}{\textbf{#1}}}
\newcommand{\BuiltInTok}[1]{#1}
\newcommand{\ExtensionTok}[1]{#1}
\newcommand{\PreprocessorTok}[1]{\textcolor[rgb]{0.56,0.35,0.01}{\textit{#1}}}
\newcommand{\AttributeTok}[1]{\textcolor[rgb]{0.77,0.63,0.00}{#1}}
\newcommand{\RegionMarkerTok}[1]{#1}
\newcommand{\InformationTok}[1]{\textcolor[rgb]{0.56,0.35,0.01}{\textbf{\textit{#1}}}}
\newcommand{\WarningTok}[1]{\textcolor[rgb]{0.56,0.35,0.01}{\textbf{\textit{#1}}}}
\newcommand{\AlertTok}[1]{\textcolor[rgb]{0.94,0.16,0.16}{#1}}
\newcommand{\ErrorTok}[1]{\textcolor[rgb]{0.64,0.00,0.00}{\textbf{#1}}}
\newcommand{\NormalTok}[1]{#1}
\usepackage{longtable,booktabs}
\usepackage{graphicx,grffile}
\makeatletter
\def\maxwidth{\ifdim\Gin@nat@width>\linewidth\linewidth\else\Gin@nat@width\fi}
\def\maxheight{\ifdim\Gin@nat@height>\textheight\textheight\else\Gin@nat@height\fi}
\makeatother
% Scale images if necessary, so that they will not overflow the page
% margins by default, and it is still possible to overwrite the defaults
% using explicit options in \includegraphics[width, height, ...]{}
\setkeys{Gin}{width=\maxwidth,height=\maxheight,keepaspectratio}
\IfFileExists{parskip.sty}{%
\usepackage{parskip}
}{% else
\setlength{\parindent}{0pt}
\setlength{\parskip}{6pt plus 2pt minus 1pt}
}
\setlength{\emergencystretch}{3em}  % prevent overfull lines
\providecommand{\tightlist}{%
  \setlength{\itemsep}{0pt}\setlength{\parskip}{0pt}}
\setcounter{secnumdepth}{5}
% Redefines (sub)paragraphs to behave more like sections
\ifx\paragraph\undefined\else
\let\oldparagraph\paragraph
\renewcommand{\paragraph}[1]{\oldparagraph{#1}\mbox{}}
\fi
\ifx\subparagraph\undefined\else
\let\oldsubparagraph\subparagraph
\renewcommand{\subparagraph}[1]{\oldsubparagraph{#1}\mbox{}}
\fi

%%% Use protect on footnotes to avoid problems with footnotes in titles
\let\rmarkdownfootnote\footnote%
\def\footnote{\protect\rmarkdownfootnote}

%%% Change title format to be more compact
\usepackage{titling}

% Create subtitle command for use in maketitle
\newcommand{\subtitle}[1]{
  \posttitle{
    \begin{center}\large#1\end{center}
    }
}

\setlength{\droptitle}{-2em}
  \title{Comparative Methods}
  \pretitle{\vspace{\droptitle}\centering\huge}
  \posttitle{\par}
  \author{Brian O'Meara}
  \preauthor{\centering\large\emph}
  \postauthor{\par}
  \predate{\centering\large\emph}
  \postdate{\par}
  \date{2017-04-15}

\usepackage{booktabs}
\usepackage{makeidx}
\makeindex

\usepackage{amsthm}
\newtheorem{theorem}{Theorem}[chapter]
\newtheorem{lemma}{Lemma}[chapter]
\theoremstyle{definition}
\newtheorem{definition}{Definition}[chapter]
\newtheorem{corollary}{Corollary}[chapter]
\newtheorem{proposition}{Proposition}[chapter]
\theoremstyle{definition}
\newtheorem{example}{Example}[chapter]
\theoremstyle{remark}
\newtheorem*{remark}{Remark}
\begin{document}
\maketitle

{
\setcounter{tocdepth}{1}
\tableofcontents
}
\chapter{Introduction}\label{introduction}

This book was created as part of my
\href{http://www.phylometh.org}{PhyloMeth} class, which focuses on
sensibly using and developing comparative methods. It will be actively
developed over the course of Spring 2017, so if you don't like this
version (see date above), check back soon! The book is available
\href{https://bookdown.org/bomeara/comparative-methods/}{here} but you
can fork it, add issues, and look at raw source code at
\url{https://github.com/bomeara/ComparativeMethodsInR}. {[}Note I'll be
changing the name of the repo eventually; the course is largely in R
(not entirely) but of course many key methods appear in other
languages.{]}

\section{Learning objectives}\label{learning-objectives}

Readers of this book will be able to:

\begin{itemize}
\tightlist
\item
  Approach a study of a group of organisms by developing meaningful
  questions
\item
  Identify the appropriate methods to answer these questions
\item
  Where methods do not yet exist, be able to work on potential new
  methods
\item
  Understand limtations of methods and how to evaluate these limits
\item
  Draw sensible biological conclusions
\end{itemize}

\section{Prerequisites}\label{prerequisites}

These are mostly prereqs for doing exercises associated with the class,
but will help readers of the book, too.

\subsection{R}\label{r}

Many methods are now implemented in R \citep{R-base}: the
\href{http://cran.r-project.org/web/views/Phylogenetics.html}{phylogenetics
task view} has a brief overview. You can also install the relevant
packages that are on CRAN and R-Forge using the task view itself:

\begin{verbatim}
install.packages("ctv")
library(ctv)
install.views("Phylogenetics")
\end{verbatim}

Note that this will not install packages that are on GitHub or authors'
individual websites. The \texttt{devtools} package can be useful for
installing packages directly from GitHub.

\subsection{Docker}\label{docker}

Another option for installing things is to use the
\href{https://hub.docker.com/r/bomeara/phydocker/}{phydocker} instance
for Docker. \href{https://www.docker.com}{Docker} is (oversimplifying)
like a very lightweight virtual machine. Note that it runs on Macs,
Linux, Windows (Pro, Enterprise, and Education versions; for other
versions, use
\href{https://docs.docker.com/toolbox/toolbox_install_windows/}{Docker
Toolbox}), and various cloud service providers (i.e., you could throw
money at Amazon to run this on one of their servers). This instance runs
a copy of RStudio Server that has most of the relevant phylogenetic
packages already installed. Once you have Docker installed, you can do

\texttt{docker\ run\ -it\ -p\ 8787:8787\ bomeara/phydocker}

to run it as an RStudio Server.

If you want to use a local folder, you can use

\texttt{docker\ run\ -it\ -v\ /Path/To/My/Folder:/data\ -p\ 8787:8787\ bomeara/phydocker}

Change \texttt{/Path/To/My/Folder} to the absolute path to the folder
you want access to (any subfolders will also be accessible). You can
read and write to this in RStudio as the \texttt{/data} directory. In
your web browser, go to \texttt{localhost:8787}, enter username and
password (both are \texttt{rstudio}), to launch a version of RStudio
that will run in your browser and have everything you might need. You
might want to do \texttt{setwd("/data")} to make sure you're in the
right directory. You can save any results or figures to this directory
and it will still exist when you quit this instance.

\subsection{Other}\label{other}

RevBayes, BEAST, RAxML, and much other key software implement important
methods in phylogenetics but are not in R. Readers will need to install
these and many more, but fortunately the authors of much of this
software have excellent tutorials already.

\chapter{First steps}\label{first-steps}

First, understand the question you want to answer. There are a wide
variety of methods, and they wax and wane in popularity, but the key to
doing good science is addressing compelling questions, not using the
latest method. Once you have that question, find the appropriate methods
(and, depending on how early it is in the study design, the right taxa
and data) to address it. Understand how those methods work, including
the various ways they can fail (as all can).

\section{Questions}\label{questions}

``The currency of science is papers.''

``You need to get grants to get a job and tenure.''

Both true (for those pursuing traditional academic careers), but it can
be easy to lose sight of the reason we do science: to learn about the
natural world. Too often, I see students and other colleagues focus on
fast ways to get high profile papers out without caring much about the
questions. Some of this takes the form of what I call
\protect\hyperlink{dull-model-testing}{dull model testing}: seeking to
reject a trivial null that no one believed in anyway (Has
diversification rate ever changed through time? Do terrestrial and
aquatic species have exactly the same body size over time?). However, it
can also be, ``I need to add something interesting to this basic
phylogeny paper to get it published -- what can I map on the tree?''
There can also be questions asked where it seems, upon reflection, that
results will not be credible (can one really estimate 999 independent
diversification rates from a single tree with 500 species?).

A better approach, and one adopted in this book, is to start from
questions that we actually learn something from answering. For example,
we believe that how flowering plants reproduce (selfing versus
outcrossing) can affect diversification rate. The first step is to ask
if that is really true and show it statistically, but that is largely
going to be a question of power: no one would really think that these
two life history strategies would lead to exactly the same speciation
rates \emph{and} exactly the same extinction rates: selfers might more
readily speciate since they can settle new areas and not lack for mates,
for example. One could publish a paper on just this using one of variety
of methods (see \protect\hyperlink{diversification}{Diversification})
and be done. However, that is a largely sterile question: are two
different things unequal? A more important question, once a difference
is shown, is what this explains about the world. For example, Igic and
Goldberg wondered why selfing persisted despite having a lower overall
diversification rate than outcrossing. In answering that more
interesting question, they found that it stemmed from subsidizing:
outcrossers diversified more quickly, but transitions from outcrossing
to selfing occurred much more frequently than the reverse: species moved
into selfing from outcrossing, but had an overall negative
diversification rate. This suggests an interesting conflict between
microevolution (factors leading to selfing, in this case), and
macroevolution (the differential diversification of species). In another
paper, we looked at floral morphological traits to see which
combinations led to higher rates of diversification (O'Meara, Smith, et
al); we found one combination had a major impact, but also discovered
that it was still fairly infrequent in flowering plants due to the
estimated tens of millions of years required to assemble this
combination from the angiosperm ancestral state. We thus learned about
how slow trait evolution can hold back diversification over a very long
time period.

Many interesting questions hinge on parameter estimation. How much worse
is it long term to be a selfer? How long will it take to evolve multiple
floral traits? How do species typically move from one habitat to
another?

\subsection{Other resources}\label{other-resources}

There are many books and articles written about phylogenetic analysis.
Some of the key books for readers of this one:

Tree Thinking

Felsenstein

Yang

Steele

An important resource for learning about phylogenetics in R is Emmanuel
Paradis' book,
\href{http://www.springer.com/us/book/9781461417422}{Analysis of
Phylogenetics and Evolution with R}. This is written by the same person
who is the lead author of the essential \texttt{ape} package
\citep{R-ape} for phylogenetics in R.

\chapter{Getting data and trees into
R}\label{getting-data-and-trees-into-r}

\section{Data and tree object types}\label{data-and-tree-object-types}

In R, there are many kinds of objects. These kinds are called
``classes''. For example, take the text string ``Darwin''.

\begin{Shaded}
\begin{Highlighting}[]
\KeywordTok{class}\NormalTok{(}\StringTok{"Darwin"}\NormalTok{)}
\end{Highlighting}
\end{Shaded}

\begin{verbatim}
## [1] "character"
\end{verbatim}

It is a \texttt{character} class. \texttt{pi} is a defined constant in
R:

\begin{Shaded}
\begin{Highlighting}[]
\KeywordTok{print}\NormalTok{(pi)}
\end{Highlighting}
\end{Shaded}

\begin{verbatim}
## [1] 3.141593
\end{verbatim}

\begin{Shaded}
\begin{Highlighting}[]
\KeywordTok{class}\NormalTok{(pi)}
\end{Highlighting}
\end{Shaded}

\begin{verbatim}
## [1] "numeric"
\end{verbatim}

Its class is \texttt{numeric} {[}and note that its value is stored with
more precision than is printed on screen{]}.

Objects can sometimes be converted from one class to another, often
using an \texttt{as.*} function:

\begin{Shaded}
\begin{Highlighting}[]
\NormalTok{example.}\DecValTok{1}\NormalTok{ <-}\StringTok{ "6"}
\KeywordTok{print}\NormalTok{(example.}\DecValTok{1}\NormalTok{)}
\end{Highlighting}
\end{Shaded}

\begin{verbatim}
## [1] "6"
\end{verbatim}

\begin{Shaded}
\begin{Highlighting}[]
\KeywordTok{class}\NormalTok{(example.}\DecValTok{1}\NormalTok{)}
\end{Highlighting}
\end{Shaded}

\begin{verbatim}
## [1] "character"
\end{verbatim}

\begin{Shaded}
\begin{Highlighting}[]
\NormalTok{example.}\DecValTok{2}\NormalTok{ <-}\StringTok{ }\KeywordTok{as.numeric}\NormalTok{(example.}\DecValTok{1}\NormalTok{)}
\KeywordTok{print}\NormalTok{(example.}\DecValTok{2}\NormalTok{)}
\end{Highlighting}
\end{Shaded}

\begin{verbatim}
## [1] 6
\end{verbatim}

\begin{Shaded}
\begin{Highlighting}[]
\KeywordTok{class}\NormalTok{(example.}\DecValTok{2}\NormalTok{)}
\end{Highlighting}
\end{Shaded}

\begin{verbatim}
## [1] "numeric"
\end{verbatim}

\begin{Shaded}
\begin{Highlighting}[]
\NormalTok{example.}\DecValTok{2} \OperatorTok{*}\StringTok{ }\DecValTok{7}
\end{Highlighting}
\end{Shaded}

\begin{verbatim}
## [1] 42
\end{verbatim}

Trying to multiply example.1 by seven results in an error: you are
trying to multiply a character string by a number, and R does not
automatically convert classes. Classes have many uses in R; for example,
one can write a different \texttt{plot()} function for each class, so
that a tree is plotted one way, while a result from a regression model
is plotted a different way, but users just have to call \texttt{plot()}
on each and R knows what to do.

In phylogenetics, we mostly care about classes for trees, for data, and
for things to hold trees and data.

\subsection{Tree classes}\label{tree-classes}

The main tree class in R is \texttt{phylo} and is defined in the
\texttt{ape} package. Let's look at one in the wild:

\begin{Shaded}
\begin{Highlighting}[]
\KeywordTok{library}\NormalTok{(ape)}
\NormalTok{phy <-}\StringTok{ }\NormalTok{ape}\OperatorTok{::}\KeywordTok{rcoal}\NormalTok{(}\DecValTok{5}\NormalTok{) }\CommentTok{#to make a random five taxon tree}
\KeywordTok{print}\NormalTok{(phy)}
\end{Highlighting}
\end{Shaded}

\begin{verbatim}
## 
## Phylogenetic tree with 5 tips and 4 internal nodes.
## 
## Tip labels:
## [1] "t4" "t1" "t2" "t5" "t3"
## 
## Rooted; includes branch lengths.
\end{verbatim}

\begin{Shaded}
\begin{Highlighting}[]
\KeywordTok{str}\NormalTok{(phy)}
\end{Highlighting}
\end{Shaded}

\begin{verbatim}
## List of 4
##  $ edge       : int [1:8, 1:2] 6 7 9 9 7 6 8 8 7 9 ...
##  $ edge.length: num [1:8] 0.8398 0.9268 0.0806 0.0806 1.0074 ...
##  $ tip.label  : chr [1:5] "t4" "t1" "t2" "t5" ...
##  $ Nnode      : int 4
##  - attr(*, "class")= chr "phylo"
##  - attr(*, "order")= chr "cladewise"
\end{verbatim}

This is the one used in most packages. However, it has some technical
disadvantages (sensitivity to internal structure, no checking of
objects) that has led to the \texttt{phylo4} format for trees and
\texttt{phylo4d} for trees plus data in the \texttt{phylobase} package.
Other packages add on to the \texttt{phylobase} format (i.e.,
\texttt{phytool}'s \texttt{simmap} format) but these are typically not
shared across packages.

\section{Sequence data}\label{sequence-data}

\section{Other character data}\label{other-character-data}

\section{Phylogenies}\label{phylogenies}

The most common way to load trees is to use \texttt{ape}'s functions:

\begin{verbatim}
phy <- ape::read.tree(file='treefile.phy')
\end{verbatim}

To get a tree in
\href{http://evolution.genetics.washington.edu/phylip/newicktree.html}{Newick
format} (sometimes called Phylip format): essentially a series of
parenthetical statements. An example (from \texttt{ape}'s documentation)
is \texttt{((Strix\_aluco:4.2,Asio\_otus:4.2):3.1,Athene\_noctua:7.3);}.
The format name comes from the name of the
\href{http://newicks.com}{lobster house} where several major
phylogenetic software developers met to agree on a tree format.

You can use the same function to enter tree strings directly, changing
the argument from the \texttt{file} containing the tree to \texttt{text}
containing the tree string:

\begin{Shaded}
\begin{Highlighting}[]
\NormalTok{phy <-}\StringTok{ }\NormalTok{ape}\OperatorTok{::}\KeywordTok{read.tree}\NormalTok{(}\DataTypeTok{text =} \StringTok{'((Strix_aluco:4.2,Asio_otus:4.2):3.1,Athene_noctua:7.3);'}\NormalTok{)}
\end{Highlighting}
\end{Shaded}

Note the trailing semicolon.

One thing that can trip users up with \texttt{read.tree()} (and the
\texttt{read.nexus()} function, below) is that the output class depends
on the input. If you read from a file with one tree, the returned output
is a single tree object with class \texttt{phylo}. You can then use
\texttt{plot()} on this object to draw the tree, pass this object into a
comparative methods package to estimate rates, and so forth. If the file
has more than one tree, the returned class is \texttt{multiphylo}:
\texttt{plot()} will automatically cycle through plots as you type
return, most comparative method implementations will fail (they are
written to expect one tree of class \texttt{phylo}, not a vector of
trees in a different class). \texttt{read.tree()} has an optional
\texttt{keep.multi} function: if set to TRUE, the class is always
\texttt{multiphylo}, and you can always get the first tree by getting
the first element in the returned object:

\begin{verbatim}
phy.multi <- ape::read.tree(file='treefile.phy', keep.multi = TRUE)
phy <- phy.multi[[1]]
\end{verbatim}

For NEXUS formatted files (Maddison et al., 2007), \texttt{ape}'s
\texttt{read.nexus()} function can pull in the trees (and its
\texttt{read.nexus.data()} function can pull in data from a NEXUS file).
NEXUS is a very flexible format, and there are valid NEXUS files that
still cause errors with \texttt{ape}'s function. A more robust function
to read in NEXUS trees is the package \texttt{phylobase}'s
\texttt{readNexus()} function (note the lack of a period and different
capitalization of Nexus from \texttt{ape}'s similar function).
\texttt{phylobase} uses a different structure to store trees than
\texttt{ape} does.

\subsection{Great scientists steal}\label{great-scientists-steal}

Scientists have been creating trait-based phylogenetic trees for
decades. These scientists are often experts in their group, in potential
problems in their data, in how to use relevant software. In other words,
their trees are likely to be better than any you make. Traditionally,
these trees are published as a figure in a paper, largely unavailable
for reuse. This hurts reproducibility, makes it less likely for the work
to be cited, and stymies scientific progress in general. However, the
field is increasingly moving to more frequent deposition of trees in
reusable form: sometimes based on author initiative, sometimes based on
journal requirements. The main repository for this is
\href{http://treebase.org}{TreeBase}: if you are reading a paper, and
want to use its tree, that's the first place to look. You can also use
their website to search for taxa. The trees can be downloaded and loaded
into R using \texttt{phylobase} (the NEXUS format used by TreeBase is
hard for \texttt{ape} to load).

Another approach that is growing in importance is
\href{http://otol.org}{Open Tree of Life}. It seeks to synthesize
thousands of trees to create a single tree of life. The \texttt{rotl}
package can download this synthetic tree or components of it (the tree
for a particular genus, for example). For most groups, however, the
synthetic tree is largely based on taxonomy, so it is not very resolved.
This is improving as the database of trees available for Open Tree's
synthesis grows (to add to it, go to
\_\_\_\_\_\_\_\_\_\_\_\_\_\_\_\_\_\_\_), but for most scientific
studies, I wouldn't currently suggest using the synthetic tree (but for
getting a sense for a group, making a tree for a class, it can be
useful; also see the \href{http://www.phylotastic.org}{Phylotastic}
project for ways to use trees in teaching or other purposes). However,
the Open Tree project also has a cache of thousands of trees that have
been hand curated (taxonomic names resolved, ingroups specified, tree
type recorded, etc.). The \texttt{rotl} package lets you download these,
too. For most analyses, you want trees with branch lengths, and so you
can download just chronograms. For example,

\begin{Shaded}
\begin{Highlighting}[]
\CommentTok{#rotl::_________________}
\end{Highlighting}
\end{Shaded}

Two important notes about reusing trees:

\begin{itemize}
\tightlist
\item
  \textbf{Give credit}: If your entire paper is based on the tree from
  one other paper, you \textbf{must} cite that paper (and also the ways
  you got the tree, including the packages and/or repositories). If it's
  based on trees from around a dozen papers, you should cite them, too.
  If you're getting into the hundreds, many editors will object to
  properly citing them all, but one compromise approach until a better
  way of giving credit appears is to have supplemental info or an
  appendix with citations for all the relevant papers (including DOIs to
  make these easier to parse later)
\item
  \textbf{Tree quality matters}: As you will see in later sections, many
  comparative methods are based on using branch lengths: look at
  different rates of character evolution, looking at diversification
  rates over time, etc. If your starting tree is wrong, even if the
  topology is perfect but the branch lengths are wrong, later downstream
  analyses are also likely to be wrong. Some methods (like independent
  contrasts) are fairly robust to this
  (\_\_\_\_\_\_\_\_\_\_\_\_\_\_\_\_\_\_), but the field has not tested
  many others yet, and most should be far more sensitive than contrasts.
  This matters less if you are testing dull hypotheses (see Chapter
  \_\_\_\_\_\_\_\_\_\_) but for folks working on biology where
  understanding processes, especially using parameter estimates, is the
  point, just taking a tree and making up branch lengths is often a bad
  idea.
\end{itemize}

\section{Reconciling datasets}\label{reconciling-datasets}

We use scientific names to communicate clearly. In the picture below
\_\_\_\_\_\_\_\_\_\_\_\_\_\_\_\_\_\_\_\_\_\_\_\_, ``Look at the robin!''
will have an American glance at the bird on the left, and a Brit look at
the bird on the right, but both, if trained sufficiently will know which
to look at if told to look at \emph{\_\_\_\_\_Scientific
name\_\_\_\_\_\_\_}. We thus use scientific names in writing. However,
the correct scientific name for a specimen can change for various
reasons:

\begin{itemize}
\tightlist
\item
  A species is split into two species: some individual specimens remain
  in the original species, others are given a new species names (rules
  of taxonomy allow this, and give constraints on how the new species
  can be named and described)
\item
  Two species are lumped into one species: some individual specimens
  thus have their names changed (and which name persists after the merge
  is specified by the rules of taxonomy)
\item
  A higher level group is changed. For example, \_\_\_\_\_\_\_\_\_
  proposed to split the \emph{Anolis} genus into eight genera. Thus the
  genus name for some species changes, and sometimes the species name
  itself changes to match the genus names: \emph{\_\_\_\_\_\_} becomes
  \emph{\_\_\_\_\_\_\_\_\_\_\_\_}. This can be a merge or a split. This
  is often motivaed by a new discovery (the group known as acacias are
  not a clade (an ancestor and all its descendants) and since we only
  want to name clades, one of the groups needs a new name).
\item
  An error is fixed. For example, it could be discovered that there was
  an earlier name for a species in the literature, and so the species
  name must be changed based on the rules of priority.
\end{itemize}

Importantly, for all but the last point, it is perfectly valid based on
the rules of taxonomy for different scientists to use the names before
and after the change.

\chapter{Visualizing data before use}\label{visualizing-data-before-use}

A key step in any analysis is looking at the data. If you have loaded
protein coding DNA sequences, are they aligned correctly? Are the codon
positions specified correctly? For trait data, is everything measured in
the same units, or are some oddly a thousand-fold higher than others?
Are you dealing with an older dataset format that uses -1 or 19 for
missing data, and have you incorrectly treated those as observations? Is
your tree ultrametric?

It is easy to overlook this step, but you can draw the wrong conclusions
based on errors at this stage. Most peer reviewers will not notice this,
either, so your error could slip into the literature and mislead others.
Take the time to get to know your data.

\hypertarget{dull-model-testing}{\chapter{Dull model
testing}\label{dull-model-testing}}

Almost all biologists believe this about the world:

\begin{itemize}
\tightlist
\item
  All species evolve identically
\item
  Rates of trait evolution are the same
\item
  Optimal states are the same
\item
  Speciation rates never change
\item
  Traits are uncorrelated
\item
  Species evolve completely independently
\item
  Extinction never happens
\item
  All evolutionary rates are constant

  \begin{itemize}
  \tightlist
  \item
    Across all time
  \item
    Across all space
  \end{itemize}
\end{itemize}

However, a scrappy group of biologists are using comparative methods to
attack the mainstream view. For example, using diversification analyses,
they can show that extinction can sometimes be greater than zero. Using
analyses of trait evolution, they have found that different species
actually have different rates of evolution: whale body mass does
\emph{not} evolve in the same way bat body mass does. These ideas are
rocking the scientific establishment.

Of course, the above is all fiction. We \emph{know} that different
things are\ldots{} different, because they're not the same. We know
about extinction, about rates changing over time, about how traits must
interact with each other. But the way we do science does not reflect
this. Instead, when doing empirical analyses, we focus on rejecting
trivial null models, or more simply, dull models. It is useful to show
that using a more complex, biologically more credible model is
warranted, but too many studies just stop there: a pure birth model is
rejected for a logistic growth model for number of species, a single
Brownian motion rate model is rejected for an Ornstein-Uhlenbeck model,
etc. However, rejecting dull models we did not believe in does not
advance science: it tells us more about the power of our study than
about actual biological mechanisms. Of course different groups have
different rates of evolution: what is the magnitude of the difference?
Getting rates with uncertainty is a better way of getting at the
biological meaning of differences.

Dull model testing comes up in discussion of a method's fitness, too.
The first question asked of a new method, or a published model under
attack, is its type I error rate. This is relevant: a method that too
often picks an alternate model when the null is true is worrisome.
However, it is also not especially relevant biologically. The null model
is never true. It may be that due to small sample size, the null is the
best-fitting model, but in any empirical scenario the true model is
never the null.

\chapter{Testing models and methods}\label{testing-models-and-methods}

\subsection{Objectives}\label{objectives}

\begin{itemize}
\tightlist
\item
  Understand distinction between model fit and model adequacy
\item
  Identify and avoid pitfalls in evaluating methods
\item
  Be able to identify methods that have been tested well.
\end{itemize}

\subsection{Model fit and accuracy}\label{model-fit-and-accuracy}

When we use models to understand biology, it helps if they are
appropriate for the data. Most importantly, this gives meaningful
parameter estimates. If the true model is one of constant
diversification rates except for a single pulse of extinction at the KT
boundary, and the data include sampling only 25\% of current diversity,
we could fit a logistic diversification model, and it could give us an
estimate of carrying capacity, perhaps even complete with uncertainty,
but the reality is there is no carrying capacity. If the question were
simply about comparing models, a test of whether a logistic or Yule
model fits the data best, we will get an answer, but it does not help us
understand reality: neither model is correct in our case.

\begin{table}

\caption{\label{tab:unnamed-chunk-8}Table of results from simulating a 2000 taxon tree under a pure birth model plus one mass extinction, then sampling tips perfectly randomly down to a 500 taxon tree.}
\centering
\begin{tabular}[t]{l|r|r|r}
\hline
  & deltaAIC & birth.rate & carrying.capacity\\
\hline
Yule & 55.772 & 0.044 & NA\\
\hline
Logistic & 0.000 & 0.063 & 1000\\
\hline
\end{tabular}
\end{table}

In the above example, the result shows that the best model is one of
logistic growth, with a carrying capacity of 1000. However, remember
that the tree used had 2000 tips to start (they were subsampled to get a
500 taxon observed tree). Neither the model nor the parameter estimate
is right, so this exercise would tell us little about biology. It
\emph{is} likely publishable.

There are thus three questions to answer when thinking about models:

\begin{enumerate}
\def\labelenumi{\arabic{enumi})}
\tightlist
\item
  Are the approximations in my models biologically reasonable?
\item
  Which model(s) fit best?
\item
  Are my models adequate?
\end{enumerate}

\chapter{Testing methods}\label{testing-methods}

\subsection{Objectives}\label{objectives-1}

\begin{itemize}
\tightlist
\item
  Identify and avoid pitfalls in evaluating methods
\item
  Be able to identify methods that have been tested well.
\end{itemize}

\subsection{Kinds of testing}\label{kinds-of-testing}

There are two kinds of testing. One can test the software to make sure
it works properly. If you are trying to calculate the average of a set
of observations, are you using \texttt{mean} or incorrectly using
\texttt{median}? Does it use all the data or does it drop anything past
the fifth observation? For this kind of question, it can be helpful to
do test driven development: write a test, then write code, and
automatically check the code to see if it passes the test. Then, as you
change code, you can rerun all the old tests to verify they still work.
This is often known as unit testing.

But even if software has correctly implemented a method, a more
compelling question is whether the method itself is any good. This comes
down to a few questions:

\subsection{Type I error}\label{type-i-error}

This is when a model incorrectly rejects a true null hypothesis. For
example, do clade A and clade B have exactly the same rate of evolution?
If the truth is that they do, rejecting that to say they are unequal is
a type I error. To test this property, data are ofen simulated under the
null, analyzed under the null and alternate hypotheses, and the
proportion of times the null is incorrectly rejected noted. For a
typical significance theshold of 0.05, this should be 5\% of the time.

This is a major focus\ldots{}

\subsection{Type II error}\label{type-ii-error}

This is incorrectly accepting a false null.

\subsection{Getting rid of typological
thinking}\label{getting-rid-of-typological-thinking}

In biology, typological thinking is bad: one of Darwin's great insights
was that there is substantial variation in nature. However, our
statistical thinking is often limited (see also chapter on dull
hypothesis testing). Appropriate Type I error rates is a nice property,
but how often is the null \emph{actually} true? Never.

\chapter{Continuous traits}\label{continuous-traits}

\section{Objectives}\label{objectives-2}

By the end of this chapter, you will:

\begin{itemize}
\tightlist
\item
  Understand various continuous trait models
\item
  Be able to run key software
\end{itemize}

Make sure to \textbf{read the relevant papers}:
\url{https://www.mendeley.com/groups/8111971/phylometh/papers/added/0/tag/week7/}

Last week we did some simulation under Brownian motion and talked about
using this model for dealing with correlations (as in independent
contrasts \citep{Felsenstein1985a}). The central limit theorem is great:
as you add changes, you converge back to a normal distribution. But what
if the changes aren't i.i.d.? For example, what if the rate of body size
evolution of birds dramatically increased once other dinosaurs went
extinct? We would have variance accumulating linearly with time before
and after the KT extinction, but the rate of increase would be different
between the two time periods.

\textbf{Do the homework at}
\url{https://github.com/PhyloMeth/ContinuousTraits}

You will: * Use Geiger to estimate rate of evolution under Brownian
motion * Figure out what the units are * Try other ways of scaling rates
* Compare different models using OUwie * Do model comparison

\chapter{Brownian Motion and
Correlations}\label{brownian-motion-and-correlations}

\emph{In progress}

\section{Objectives}\label{objectives-3}

By the end of this chapter, you will:

\begin{itemize}
\tightlist
\item
  Understand the importance of dealing with correlations in an
  evolutionary manner
\item
  Know methods for looking at correlations of continuous and discrete
  traits
\item
  Be able to point to reasons to be concerned.
\end{itemize}

Make sure to \textbf{read the relevant papers}:
\url{https://www.mendeley.com/groups/8111971/phylometh/papers/added/0/tag/week6/}

\section{Brownian motion}\label{brownian-motion}

First, let's get a tree:

\begin{Shaded}
\begin{Highlighting}[]
\KeywordTok{library}\NormalTok{(rotl)}
\KeywordTok{library}\NormalTok{(ape)}
\NormalTok{phy <-}\StringTok{ }\KeywordTok{get_study_tree}\NormalTok{(}\StringTok{"ot_485"}\NormalTok{, }\StringTok{"tree1"}\NormalTok{)}
\KeywordTok{plot}\NormalTok{(phy, }\DataTypeTok{cex=}\FloatTok{0.5}\NormalTok{)}
\KeywordTok{axisPhylo}\NormalTok{(}\DataTypeTok{backward=}\OtherTok{TRUE}\NormalTok{)}
\end{Highlighting}
\end{Shaded}

\includegraphics{comparative-methods_files/figure-latex/unnamed-chunk-9-1.pdf}

Note that this tree is a chronogram.

Let's simulate data on this tree. But what model to use? For now, let's
assume we are looking at continuous traits, things like body size. Over
evolutionary time, these probably undergo a series of changes that then
get added up. A species has an average mass of 15 kg, then it goes to
15.1 kg, then 14.8 kg, and so forth. But how could those changes be
distributed?

Start with a uniform distribution. Take a starting value of 0, then pick
a number from -1 to 1 to add to it (in other words,
\texttt{runif(n=1,\ min=-1,\ max=1)}). There are efficient ways to do
this for many generations, but let's do the obvious way: a simple
\texttt{for} loop. Do it for 100 generations.

\begin{Shaded}
\begin{Highlighting}[]
\NormalTok{ngen <-}\StringTok{ }\DecValTok{100}
\NormalTok{positions <-}\StringTok{ }\KeywordTok{c}\NormalTok{(}\DecValTok{0}\NormalTok{, }\KeywordTok{rep}\NormalTok{(}\OtherTok{NA}\NormalTok{,ngen))}
\ControlFlowTok{for}\NormalTok{ (i }\ControlFlowTok{in} \KeywordTok{sequence}\NormalTok{(ngen)) \{}
\NormalTok{  positions[i}\OperatorTok{+}\DecValTok{1}\NormalTok{] <-}\StringTok{ }\NormalTok{positions[i] }\OperatorTok{+}\StringTok{ }\KeywordTok{runif}\NormalTok{(}\DecValTok{1}\NormalTok{,}\OperatorTok{-}\DecValTok{1}\NormalTok{,}\DecValTok{1}\NormalTok{)}
\NormalTok{\}}
\KeywordTok{plot}\NormalTok{(}\DataTypeTok{x=}\NormalTok{positions, }\DataTypeTok{y=}\KeywordTok{sequence}\NormalTok{(}\KeywordTok{length}\NormalTok{(positions)), }\DataTypeTok{xlab=}\StringTok{"trait value"}\NormalTok{, }\DataTypeTok{ylab=}\StringTok{"generation"}\NormalTok{, }\DataTypeTok{bty=}\StringTok{"n"}\NormalTok{, }\DataTypeTok{type=}\StringTok{"l"}\NormalTok{)}
\end{Highlighting}
\end{Shaded}

\includegraphics{comparative-methods_files/figure-latex/unnamed-chunk-10-1.pdf}

We can repeat this simulation many times and see what the pattern looks
like:

\begin{Shaded}
\begin{Highlighting}[]
\NormalTok{ngen <-}\StringTok{ }\DecValTok{100}
\NormalTok{nsims <-}\StringTok{ }\DecValTok{500}
\NormalTok{final.positions <-}\StringTok{ }\KeywordTok{rep}\NormalTok{(}\OtherTok{NA}\NormalTok{, nsims)}
\CommentTok{# make a plot to hold our lines}
\KeywordTok{plot}\NormalTok{(}\DataTypeTok{x=}\KeywordTok{c}\NormalTok{(}\OperatorTok{-}\DecValTok{1}\NormalTok{,}\DecValTok{1}\NormalTok{)}\OperatorTok{*}\NormalTok{ngen, }\DataTypeTok{y=}\KeywordTok{c}\NormalTok{(}\DecValTok{1}\NormalTok{, }\DecValTok{1}\OperatorTok{+}\NormalTok{ngen), }\DataTypeTok{xlab=}\StringTok{"trait value"}\NormalTok{, }\DataTypeTok{ylab=}\StringTok{"generation"}\NormalTok{, }\DataTypeTok{bty=}\StringTok{"n"}\NormalTok{, }\DataTypeTok{type=}\StringTok{"n"}\NormalTok{)}
\ControlFlowTok{for}\NormalTok{ (sim.index }\ControlFlowTok{in} \KeywordTok{sequence}\NormalTok{(nsims)) \{}
\NormalTok{  positions <-}\StringTok{ }\KeywordTok{c}\NormalTok{(}\DecValTok{0}\NormalTok{, }\KeywordTok{rep}\NormalTok{(}\OtherTok{NA}\NormalTok{,ngen))}
  \ControlFlowTok{for}\NormalTok{ (i }\ControlFlowTok{in} \KeywordTok{sequence}\NormalTok{(ngen)) \{}
\NormalTok{    positions[i}\OperatorTok{+}\DecValTok{1}\NormalTok{] <-}\StringTok{ }\NormalTok{positions[i] }\OperatorTok{+}\StringTok{ }\KeywordTok{runif}\NormalTok{(}\DecValTok{1}\NormalTok{,}\OperatorTok{-}\DecValTok{1}\NormalTok{,}\DecValTok{1}\NormalTok{)}
\NormalTok{  \}}
  \KeywordTok{lines}\NormalTok{(positions, }\KeywordTok{sequence}\NormalTok{(}\KeywordTok{length}\NormalTok{(positions)), }\DataTypeTok{col=}\KeywordTok{rgb}\NormalTok{(}\DecValTok{0}\NormalTok{,}\DecValTok{0}\NormalTok{,}\DecValTok{0}\NormalTok{,}\FloatTok{0.1}\NormalTok{))}
\NormalTok{  final.positions[sim.index] <-}\StringTok{ }\NormalTok{positions[}\KeywordTok{length}\NormalTok{(positions)]}
\NormalTok{\}}
\end{Highlighting}
\end{Shaded}

\includegraphics{comparative-methods_files/figure-latex/unnamed-chunk-11-1.pdf}

Well, that may seem odd: we're adding a bunch of uniform random values
between -1 and 1 (so, a flat distribution) and we get something that
definitely has more lines ending up in the middle than further out. Look
just at the distribution of final points:

\begin{Shaded}
\begin{Highlighting}[]
\KeywordTok{plot}\NormalTok{(}\KeywordTok{density}\NormalTok{(final.positions), }\DataTypeTok{col=}\StringTok{"black"}\NormalTok{, }\DataTypeTok{bty=}\StringTok{"n"}\NormalTok{)}
\end{Highlighting}
\end{Shaded}

\includegraphics{comparative-methods_files/figure-latex/unnamed-chunk-12-1.pdf}

Which looks almost normal. Ok, let's try a weird distribution:

\begin{Shaded}
\begin{Highlighting}[]
\NormalTok{rweird <-}\StringTok{ }\ControlFlowTok{function}\NormalTok{() \{}
\NormalTok{  displacement <-}\StringTok{ }\DecValTok{0}
  \ControlFlowTok{if}\NormalTok{(}\KeywordTok{runif}\NormalTok{(}\DecValTok{1}\NormalTok{,}\OperatorTok{-}\DecValTok{2}\NormalTok{,}\DecValTok{2}\NormalTok{) }\OperatorTok{<}\StringTok{ }\NormalTok{.}\DecValTok{1}\NormalTok{) \{}
\NormalTok{    displacement <-}\StringTok{ }\KeywordTok{rnorm}\NormalTok{(}\DecValTok{1}\NormalTok{, }\DecValTok{7}\NormalTok{, }\DecValTok{3}\NormalTok{) }\OperatorTok{+}\StringTok{ }\KeywordTok{runif}\NormalTok{(}\DecValTok{1}\NormalTok{,}\DecValTok{0}\NormalTok{,}\DecValTok{7}\NormalTok{)}
\NormalTok{  \} }\ControlFlowTok{else}\NormalTok{ \{}
\NormalTok{    displacement <-}\StringTok{ }\FloatTok{0.5} \OperatorTok{*}\StringTok{ }\KeywordTok{rexp}\NormalTok{(}\DecValTok{1}\NormalTok{, }\FloatTok{0.3}\NormalTok{) }\OperatorTok{-}\StringTok{ }\DecValTok{1}
\NormalTok{  \}}
\NormalTok{  displacement <-}\StringTok{ }\NormalTok{displacement }\OperatorTok{+}\StringTok{ }\KeywordTok{round}\NormalTok{(}\KeywordTok{runif}\NormalTok{(}\DecValTok{1}\NormalTok{,}\DecValTok{1}\NormalTok{,}\DecValTok{100}\NormalTok{) }\OperatorTok\StringTok{ }\DecValTok{7}\NormalTok{)}
  \KeywordTok{return}\NormalTok{(displacement)  }
\NormalTok{\}}

\KeywordTok{plot}\NormalTok{(}\KeywordTok{density}\NormalTok{(}\KeywordTok{replicate}\NormalTok{(}\DecValTok{100000}\NormalTok{, }\KeywordTok{rweird}\NormalTok{())), }\DataTypeTok{bty=}\StringTok{"n"}\NormalTok{)}
\end{Highlighting}
\end{Shaded}

\includegraphics{comparative-methods_files/figure-latex/unnamed-chunk-13-1.pdf}

When we ask \texttt{rweird()} for a number it sometimes gives us a
normally distributed number multiplied by a unifor distribution, other
times it gives us an exponentially distributed number, and then adds the
remainder that comes when you divide a random number by 7. So, not
exactly a simple distribution like uniform, normal, or Poisson. So,
repeating the simulation above but using this funky distribution:

\begin{Shaded}
\begin{Highlighting}[]
\NormalTok{ngen <-}\StringTok{ }\DecValTok{100}
\NormalTok{nsims <-}\StringTok{ }\DecValTok{500}
\NormalTok{final.positions <-}\StringTok{ }\KeywordTok{rep}\NormalTok{(}\OtherTok{NA}\NormalTok{, nsims)}
\CommentTok{# make a plot to hold our lines}
\KeywordTok{plot}\NormalTok{(}\DataTypeTok{x=}\KeywordTok{c}\NormalTok{(}\OperatorTok{-}\DecValTok{100}\NormalTok{,}\DecValTok{1200}\NormalTok{), }\DataTypeTok{y=}\KeywordTok{c}\NormalTok{(}\DecValTok{1}\NormalTok{, }\DecValTok{1}\OperatorTok{+}\NormalTok{ngen), }\DataTypeTok{xlab=}\StringTok{"trait value"}\NormalTok{, }\DataTypeTok{ylab=}\StringTok{"generation"}\NormalTok{, }\DataTypeTok{bty=}\StringTok{"n"}\NormalTok{, }\DataTypeTok{type=}\StringTok{"n"}\NormalTok{)}
\ControlFlowTok{for}\NormalTok{ (sim.index }\ControlFlowTok{in} \KeywordTok{sequence}\NormalTok{(nsims)) \{}
\NormalTok{  positions <-}\StringTok{ }\KeywordTok{c}\NormalTok{(}\DecValTok{0}\NormalTok{, }\KeywordTok{rep}\NormalTok{(}\OtherTok{NA}\NormalTok{,ngen))}
  \ControlFlowTok{for}\NormalTok{ (i }\ControlFlowTok{in} \KeywordTok{sequence}\NormalTok{(ngen)) \{}
\NormalTok{    positions[i}\OperatorTok{+}\DecValTok{1}\NormalTok{] <-}\StringTok{ }\NormalTok{positions[i] }\OperatorTok{+}\StringTok{ }\KeywordTok{rweird}\NormalTok{()}
\NormalTok{  \}}
  \KeywordTok{lines}\NormalTok{(positions, }\KeywordTok{sequence}\NormalTok{(}\KeywordTok{length}\NormalTok{(positions)), }\DataTypeTok{col=}\KeywordTok{rgb}\NormalTok{(}\DecValTok{0}\NormalTok{,}\DecValTok{0}\NormalTok{,}\DecValTok{0}\NormalTok{,}\FloatTok{0.1}\NormalTok{))}
\NormalTok{  final.positions[sim.index] <-}\StringTok{ }\NormalTok{positions[}\KeywordTok{length}\NormalTok{(positions)]}
\NormalTok{\}}
\end{Highlighting}
\end{Shaded}

\includegraphics{comparative-methods_files/figure-latex/unnamed-chunk-14-1.pdf}

And now let's look at final positions again:

\begin{Shaded}
\begin{Highlighting}[]
\KeywordTok{plot}\NormalTok{(}\KeywordTok{density}\NormalTok{(final.positions), }\DataTypeTok{col=}\StringTok{"black"}\NormalTok{, }\DataTypeTok{bty=}\StringTok{"n"}\NormalTok{)}
\end{Highlighting}
\end{Shaded}

\includegraphics{comparative-methods_files/figure-latex/unnamed-chunk-15-1.pdf}

Again, it looks pretty much like a normal distribution. You can try with
your own wacky distribution, and this will almost always happen (as long
as the distribution has finite variance).

Why?

Well, think back to stats: why do we use the normal distribution for so
much?

Answer: the central limit theorem. The sum (or, equivalently, average)
of a set of numbers pulled from distributions that each have a finite
mean and finite variance will approximate a normal distribution. The
numbers could all be independent and come from the same probability
distribution (i.e., could take numbers from the same Poisson
distribution), but this isn't required.

Biologically, the technical term for this is awesome. We know something
like a species mean changes for many reasons: chasing an adaptive peak
here, drifting there, mutation driving a it this way or that, etc. If
there are enough shifts, where a species goes after many generations is
normally distributed. For two species, there's one normal distribution
for their evolution from the origin of life (or the start of the tree
we're looking at) to their branching point (so they have identical
history up to then) then each evolves from that point independently
(though of course in reality they may interact; the method that's part
of the grant supporting this course allows for this). So they have
covariance due to the shared history, then accumulate variance
independently after the split. We thus use a multivariate normal for
multiple species on the tree (for continuous traits), but it again is
due to Brownian motion. \textbf{This mixture of independent and shared
evolution is quite important: it explains why species cannot be treated
as independent data points, necessitating the correlation methods that
use a phylogeny in this week's lessons.}

However, in biological data there are (at least) two issues. One is that
in some ways a normal distribution is weird: it says that for the trait
of interest, there's a positive probability for any value from negative
infinity to positive infinity. ``Endless forms most beautiful and most
wonderful have been, and are being, evolved'' {[}Darwin{]} but nothing
is so wonderful as to have a mass of -15 kg (or, for that matter, 1e7
kg). Under Brownian motion, we expect a displacement of 5 g to have
equal chance no matter what the starting mass, but in reality a shrew
species that has an average mass of 6 g is less likely to lose 5 g over
one million years than a whale species that has an average adult mass of
100,000,000 g. Both difficulties go away if we think of the
displacements not coming as an addition or subtraction to a species'
state but rather a multiplying of a state: the chance of a whale or a
shrew increasing in mass by 1\% per million years may be the same, even
if their starting mass magnitudes are very different. Conveniently, this
also prevents us getting zero or lower for a mass (or other trait being
examined). This works if we use the \textbf{log of the species trait and
treat that as evolving under Brownian motion}, and this is why traits
are commonly transformed in this way in phylogenetics (as well they
should be).

The other issue is that the normal approximation might not hold. For
example, if species are being pulled back towards some fixed value, the
net displacement is not a simple sum of the displacements: we keep
getting pulled back, in effect eroding the influence of movements the
deeper they are in the past: thus the utility of Ornstein-Uhlenbeck
models. There may also be a set of displacements that all come from one
model, then a later set of displacements that all come from some
different model: we could better model evolution, especially correlation
between species, by using these two (or more) models rather than assume
the same normal distribution throughout time: thus the utility of
approaches that allow different parameters or even different models on
different parts of the tree.

\section{Correlation}\label{correlation}

For this week, bring your data and a tree for those taxa. Fork
\url{https://github.com/PhyloMeth/Correlation} and then add scripts
there. When you're done, do a pull request. Note if you add data to that
directory and commit it, it'll be uploaded to public GitHub. Probably
not a big deal, unless you want to keep your data secret and safe (Lord
of The Rings reference; c.f. \texttt{phangorn} package).

\textbf{Do independent contrasts} using \texttt{pic()} in \texttt{ape}.
Remember to 1) positivize the contrasts (this is \emph{not} the same as
doing \texttt{abs()}). From the Garland et al. paper, think about ways
to see if there are any problems. How do contrasts affect the
correlations?

\textbf{Do Pagel94} There are at least three ways to do this in R: in
the \texttt{phytools}, \texttt{diversitree}, and \texttt{corHMM}
packages. With \texttt{phytools}, it's pretty simple: use the
\texttt{fitPagel()} function. With the others, you have to specify the
constraint matrices (this allows you to do Pagel-style tests but on a
wider range of models). Think about what you should assume at the root
state: canonical Pagel94 assumes equal probabilities of each state at
the root, but that might be a bad assumption for your taxa.

\textbf{Use another correlation method} Perhaps phyloGLM? Use the
\texttt{phylolm} package, or some other approach to look at
correlations.

\chapter{Discrete Traits}\label{discrete-traits}

\section{Objectives}\label{objectives-4}

By the end of this chapter, you will:

\begin{itemize}
\tightlist
\item
  Understand how to incorporate rate heterogeneity in discrete trait
  models
\item
  Be able to explain how to test hypotheses about univariate trait
  evolution.
\end{itemize}

Many traits can be thought of as discrete traits: a DNA site comes in
ATGC, protein have one of 20 amino acids, some animals have functional
eyes and others do not, some plants are woody and others are herbaceous.
This is nearly always an approximation. Think of something like limbs:
they seem distinct enough that we even name some groups by their count:
tetrapods, hexapods. Except that when we look closely enough, it becomes
fuzzy: insect mouthparts are derived from limbs, for example, so should
we count these highly modified limbs as limbs (and if not, where in
evolution have they become sufficiently modified to no longer count? And
are nymphalid butterflies tetrapods under that definition yet?). Are
modern whales thought to have four limbs, even though two are extremely
vestigial? Often for neontologists problematic organisms with
intermediate counts are conveniently extinct (so long,
\emph{Basilosaurus}), so we can ignore this fuzziness, but it is often
there (and paleontologists are confronted with it more often). Think
about the details of a species changing from one discrete state to
another, even for something like a seemingly perfectly discrete
character like a base changing from an A to a T. At first this is
present in just a single individual (for a multicellular diploid, on one
DNA strand in one cell in the germ line). Even if under selection, it
will take generations to sweep through to fixation: during that time,
what is ``the'' state of the species? It is even harder to discretize
characters like woodiness (how much wood is required?), eyes (when does
a fish population evolving in a cave finally ``lose'' its eyes?),
biogeography (how finely do you divide the range: by continent? biome?
state?), and so forth.

As for many decisions, this comes back to the biological hypotheses
being tested and the size of the study. For example, one question could
be does a complex trait like wings ever re-evolve once lost?
\citet{Whiting2003} examined this in stick insects: some species have
wings in both sexes, some in one only, and some lack wings in both
sexes. If the question hinges on whether loss of wing genes in a species
prevents re-evolution, then as long as one sex in a species has wings
the species should be coded as having wings. If the question hinges on
the effect of loss of wings on ability to settle new areas, it could be
that having either sex lack wings is enough to prevent effective
colonization, and thus a species with only one sex with wings should be
coded as being wingless. If the study system is large enough to have
sufficient power, one could code this as a four state character,
instead: \textbf{A}: both males and females have wings; \textbf{B}:
males have, females lack wings; \textbf{C}: males lack, females have
wings; and \textbf{D}: both males and females lack wings.

One way to deal with this is to gather discrete data as finely divided
as reasonable and then aggregate. For example, in the stick insect
example, code it as a four state character as above and then, depending
on the biological question, group them. If the question is whether wings
can reappear after being entirely lost, for example, one would group
\textbf{A}, \textbf{B}, and \textbf{C} as having wings (in at least some
members of the species, so the genes remain under selection for
functionality) and \textbf{D} as wingless, but for the dispersal
question one could lump \textbf{B}, \textbf{C}, and \textbf{D}, leaving
\textbf{A} as the other state, or even lump \textbf{B} and \textbf{C}
only.

\begin{tabular}{l|l|l|r|r|r}
\hline
males & females & four\_states & question\_1 & question\_2a & question\_2b\\
\hline
wings & wings & A & 1 & 1 & 2\\
\hline
wings & wingless & B & 0 & 1 & 1\\
\hline
wingless & wings & C & 0 & 1 & 1\\
\hline
wingless & wingless & D & 0 & 0 & 0\\
\hline
\end{tabular}

But, let's assume we can discretize traits and carry on. The simplest
discretization is binary: 0 or 1, often absence or presence (but could
be yellow or blue, etc.). Most models are like our commonly used DNA
models: continuous time with discrete changes, using the same rates
throughout. It is like a model for when an autonomous car will have an
accident: assuming the car works perfectly (gives a whole new meaning to
``blue screen of death'') there's still a chance that at some point a
human is going to run into it. There's a per hour chance of an accident:
let's assume in each hour there's a 0.03\% chance of our autonomous car
having an accident (very roughly based on Google's experience, assuming
a 40 MPH average speed). So the probability of having no accident in the
first hour of driving is 99.97\%; the probability of having no accidents
in the first 40 hours of driving is 99.97\% \^{} 40 = 98.8\%. The number
of accidents is Poisson-distributed; the wait time between accidents is
exponentially distributed. This is the model commonly used in
phylogenetics for discrete traits, though sometimes with more
complexity: one could move (with some rate) between two different rates,
as in a covarion model, for example. A very different model is
Felsenstein's threshold model, which we will discuss in a few weeks. For
now, though, just envision models with a fixed rate of change between
states as long as other characters don't change; it's possible, though,
that the state of other characters do affect these rates (which is what
correlation tests investigate). For example, the probability of
switching from clawed feet to flippers for forelimbs is probably much
higher for species that live in water than on land.

(btw, note the spelling here: having one, two or eight eyes is a
\emph{discrete} trait: individually separate and distinct. Forming an
enclosed bower for hidden mating is a \emph{discreet} trait. The former
is generally far more biologically relevant)

\hypertarget{diversification}{\chapter{Diversification}\label{diversification}}

\section{Objectives}\label{objectives-5}

By the end of this chapter, you will:

\begin{itemize}
\tightlist
\item
  Understand diversification models that don't incorporate traits
\item
  Be able to estimate diversification parameters for your data
\end{itemize}

Make sure to \textbf{read the relevant papers}:
\url{https://www.mendeley.com/groups/8111971/phylometh/papers/added/0/tag/diversification/}

And \textbf{do the relevant exercise}:
\url{https://github.com/PhyloMeth/Diversification}

\chapter{SSE methods}\label{sse-methods}

\section{Objectives}\label{objectives-6}

By the end of this chapter, you will:

\begin{itemize}
\tightlist
\item
  Understand models that look at the effects of traits on
  diversification
\item
  Understand some of the problems with these
\item
  Be able to categorize Type I and Type II errors and talk about their
  relevance.
\end{itemize}

Make sure to \textbf{read the relevant papers}:
\url{https://www.mendeley.com/groups/8111971/phylometh/papers/added/0/tag/sse/}

At this point in the course, you should be familiar with working through
getting software running. If you want more practice, you could use the
\href{https://cran.r-project.org/web/packages/hisse/vignettes/hisse-vignette.html}{vignette}
written by Jeremy Beaulieu for hisse; running diversitree is similar.
However, I think a bigger thing to discuss is how we understand a method
is working well, and how we assess whether results are believable. This
is especially prominent in discussions of diversification. One great
example of this discussion in the literature:

\begin{itemize}
\tightlist
\item
  \href{http://rstb.royalsocietypublishing.org/content/344/1307/77}{Nee
  et al. (1994)} ``Extinction rates can be estimated from molecular
  phylogenies''
\item
  \href{http://onlinelibrary.wiley.com/doi/10.1111/j.1558-5646.2009.00926.x/abstract}{Rabosky
  (2009)} ``Extinction rates should not be estimated from molecular
  phylogenies''
\item
  \href{http://onlinelibrary.wiley.com/doi/10.1111/evo.12614/abstract}{Beaulieu
  and O'Meara (2015)} ``Extinction rates can be estimated from
  moderately sized molecular phylogenies''
\item
  \href{http://onlinelibrary.wiley.com/doi/10.1111/evo.12820/full}{Rabosky
  (2015)} ``Challenges in the estimation of extinction from molecular
  phylogenies: A response to Beaulieu and O'Meara''
\end{itemize}

And that probably isn't the last word on this (no, I'm not planning
something at the moment). However, despite all the hemming and hawing
over what people \emph{should} do, people still keep using methods:
folks using \texttt{diversitree}, \texttt{BAMM}, \texttt{geiger}, or
many other approaches merrily estimate extinction rates as a necessary
part of their analyses, though they usually don't focus on them in their
studies, instead focusing on diversification (speciation - extinction)
or speciation alone. (Note: it is possible to do diversification
approaches without estimating extinction rates: one could fix extinction
rate at a known value (perhaps using the average duration of fossil
``species'' to get a fixed estimate), though what is usually done is to
fix extinction at 0 (this is called a Yule model, if speciation is
constant). This is a bit weird when you think about it: one of the few
things we truly know in biology without a doubt is that extinction is
far, far from zero in general (though this wasn't
\href{http://www.ucmp.berkeley.edu/mammal/artio/irishelk.html}{really
discovered in Western science until the 19th century})). But as
skeptical scientists, we should aim higher. It's often tempting,
especially as students or postdocs facing a difficult job market, to
focus on what we can get out: is there an exciting result that can get
past peer review and build our fame (maybe, in our tiny circle) and
fortune (ha!). But it's important to take a step back: are we confident
enough, truly (not just based on the p-value) that our results are
actually discoveries about nature? Diversification is an area (ancestral
state estimation is another) where the reality of results is especially
worrisome. Take, for example,
\href{http://onlinelibrary.wiley.com/doi/10.1111/2041-210X.12565/abstract}{Etienne
et al. (2016)}, who as part of a broader simulation study, analyzed a
dataset of 25 \emph{Dendroica} warbler species, which is a classic
dataset for these studies, fitting a logistic growth model. They found
that depending on how the model was conditioned\footnote{Conditional
  probability is the probability of an event given that some other event
  has occurred. For example, you could use past information from your
  department to estimate \texttt{Prob(getting\ tenure)}, but it is
  different if you use information that another event has occurred:
  \texttt{Prob(getting\ tenure\ \textbar{}\ made\ major\ discovery\ in\ evolution)}
  is different from
  \texttt{Prob(getting\ tenure\ \textbar{}\ four\ years\ since\ last\ publication)}.
  In this domain (diversification alone and diversification plus trait
  models), we condition on actually having a tree to look at: if the
  true model is speciation rate equals extinction rate, there's a good
  chance that most clades starting X million years ago will have gone
  extinct, so the ones we see diversified unusually quickly, and this
  has to be taken into account. The example I usually use for this is
  the idea that dolphins rescue drowning sailors. It's known dolphins
  push interesting objects in the ocean. We could interview previously
  drowning sailors that dolphins pushed towards shore, and they'll all
  say that dolphins saved them, but it's very hard to interview sailors
  the dolphins pushed the other way. As always, Randall Munroe's XKCD
  explains conditional probability best
  \includegraphics{images/seashell.png}}, something most users would
ignore, the estimated carrying capacity (in this model, the number of
warbler species at which speciation equals extinction; in other models,
this would be the number of species at which speciation is zero, with
extinction \emph{always} set to zero) for warblers is 24.59 or 6.09 or
0.656 species. That means, depending on how one conditions, we should
expect the number of \emph{Dendroica} warblers to stay exactly where it
is, crash to around six species, or crash to half a species {[}and of
course, in the latter two cases, stochastic change in number of species
will lead to them hitting zero species fairly quickly, which is an
absorbing state{]}. Given the sensitivity of the model to factors like
conditioning, any result has to be taken with a great deal of
skepticism.

For trait based models, the same history of finding problems and the
solutions (or partial solutions) has arisen (there is a paper coming out
in the \emph{American Journal of Botany} in May 2016 that goes into this
in more detail). The most relevant parts:

\href{http://onlinelibrary.wiley.com/doi/10.1111/j.0014-3820.2006.tb00517.x/abstract}{Maddison
(2006)} (and there were similar points made by others earlier) showed
that transition rates and diversification rates can be hard to
distinguish. Oversimplifying a bit, but if the rate of going from state
0 to state 1 is higher than the reverse rate, then over time there
should be many more taxa in state 1 than 0. If the diversification rate
in state 1 is higher than in state 0, there will tend to be more taxa in
state 1 than 0. So if you see a tree with more state 1 than 0, is it
higher transition rate to 1, or higher diversification in 1? Or is it
lower transition to 1 but a high enough diversification rate that it
overwhelms it? Or just chance? The paper identified the problem but not
the solution: the last line of the abstract is ``Studies of biased
diversification and biased character change need to be unified by joint
models and estimation methods, although how successfully the two
processes can be teased apart remains to be seen.''

This was followed up by
\href{http://sysbio.oxfordjournals.org/content/56/5/701.short}{Maddison
et al. (2007)} who figured out a method to deal with this: BiSSE. And
systematists looked at it, and it was good. There is now a bestiary of
similar *SSE models for different kinds of data: QuaSSE, GeoSSE, MuSSE,
ClaSSE, and more.

However, there are concerns. The original paper showed that estimating
extinction was hard to do accurately, and other papers (Davis et al,
2013) showed that you needed a hundreds of species to find significant
results (sorry, \emph{Dendroica} -- you will forever remain a mystery).
However, a particular scary paper (and talk at Evolution, where it was
presented) was
\href{https://sysbio.oxfordjournals.org/content/early/2014/10/23/sysbio.syu070.full}{Maddison
\& FitzJohn (2014)}. Their paper mostly discussed correlated characters,
but has relevance to SSE approaches. If you have a single change on a
tree, you don't know if a higher rate in the clade descended from that
branch is due to that trait, some other trait changing on that branch,
or some other trait changing on the tree but in a way that makes it
mostly on one part of the tree.
\href{http://sysbio.oxfordjournals.org/content/early/2015/01/18/sysbio.syu131}{Rabosky
\& Goldberg (2015)} showed that the way many people interpret BiSSE
results, as a testing of a null hypothesis, can be misled if part of the
null hypothesis is wrong but not the part you're interested in.
\href{http://sysbio.oxfordjournals.org/content/early/2016/03/25/sysbio.syw022.abstract}{Beaulieu
\& O'Meara (2016)} address some of the Maddison \& FitzJohn issues (a
character changing elsewhere on the tree driving a result) but not all
the issues (a single change being sufficient to give substantial support
for an idea that that trait is driving diversification).

\subsection{Discussion prompts}\label{discussion-prompts}

\begin{enumerate}
\def\labelenumi{\arabic{enumi}.}
\tightlist
\item
  What is Type I and Type II error?
\item
  Do we care about them? Why or why not?
\item
  Compare and contrast

  \begin{itemize}
  \tightlist
  \item
    model selection
  \item
    null hypothesis rejection
  \item
    multimodel inference
  \item
    parameter estimation
  \end{itemize}
\item
  What is a good null model for trait diversification?
\item
  What is the model we use?
\item
  How do you know a method is good enough to use?

  \begin{itemize}
  \tightlist
  \item
    in general
  \item
    for \emph{your} data
  \end{itemize}
\item
  Given the controversies about diversification methods, are you willing
  to use them? Defend your view!
\end{enumerate}

For a lot of these questions, there isn't a right answer (at least, not
one I know, and certainly not an agreement in the field). But it's worth
thinking about as you develop your research career.

\chapter{RAxML}\label{raxml}

\section{Objectives}\label{objectives-7}

By the end of this week, you will:

\begin{itemize}
\tightlist
\item
  Have RAxML installed
\item
  Be able to do an analysis with likelihood with various models
\item
  Understand partitioning
\item
  Be able to use a variety of character types
\end{itemize}

RAxML (Stamatakis, 2014) is a very popular program for inferring
phylogenies using likelihood, though there are many others. It is
notable for being able to infer trees for tens of thousands of species
or more. New versions can use DNA, amino acid, SNP, and/or morphological
characters.

\section{Install RAxML}\label{install-raxml}

To begin, \textbf{install RAxML}. Follow the instructions in Step 1 of
\url{http://sco.h-its.org/exelixis/web/software/raxml/hands_on.html}.
For the fewest issues, just do \texttt{make\ -f\ Makefile.gcc} on the
command line (not in R) to compile the basic vanilla version. For actual
work, you'll likely find the versions with SSE3 and/or PTHREADS will
work faster. On a Mac (Linux is similar; RAxML has
\href{https://github.com/stamatak/standard-RAxML/tree/master/WindowsExecutables_v8.2.4}{binaries}),
the easiest way to get use this would be:

\begin{verbatim}
git clone git@github.com:stamatak/standard-RAxML.git
cd standard-RAxML
make -f Makefile.gcc
\end{verbatim}

If compiling went correctly, you should see a line like

\begin{verbatim}
gcc  -o raxmlHPC axml.o  optimizeModel.o multiple.o searchAlgo.o topologies.o parsePartitions.o treeIO.o models.o bipartitionList.o rapidBootstrap.o evaluatePartialGenericSpecial.o evaluateGenericSpecial.o newviewGenericSpecial.o makenewzGenericSpecial.o   classify.o fastDNAparsimony.o fastSearch.o leaveDropping.o rmqs.o rogueEPA.o ancestralStates.o  mem_alloc.o  eigen.o -lm
\end{verbatim}

\textbf{Now you need to put the program in a path.} This is where your
computer looks for programs to run. If you type a program name, like
\texttt{ls} or \texttt{raxmlHPC}, your computer checks the folders
indicated in the path for a program of this name; when it finds one, it
runs that. You can see your path by typing \texttt{echo\ \$PATH}. If you
want to run a program, like the newly compiled \texttt{raxmlHPC}, you
have two options: you can specify where it is each time you want to run
it, or you can put it in a folder in your existing path. The former
becomes a pain, so I'd recommend the latter. \texttt{/usr/bin} is in
your path, but this is reserved for programs your computer needs to run
-- don't mess with it. I'd suggest putting it in
\texttt{/usr/local/bin}. To do this, type

\begin{verbatim}
sudo cp raxmlHPC /usr/local/bin/raxmlHPC
\end{verbatim}

\texttt{sudo} means superuser do. It's a very powerful command.
Generally, typing on the command line you can delete files that are
important to you, but it's hard to utterly destroy your computer; with
superuser abilities, you could delete key files.

\begin{figure}
\centering
\includegraphics{images/sandwich.png}
\caption{Sudo sandwich from xkcd}
\end{figure}

Ok, so we now have RAxML installed. To run it, you could use the very
handy \texttt{ips} package to call it from R, but it doesn't have an
interface to all of the relevant commands. Instead, we're going to just
create some commands to run ourselves.

First, we need sample data sets. We will be using ones, modified
somewhat, from
\href{http://sco.h-its.org/exelixis/web/software/raxml/hands_on.html}{this
tutorial}. The original files are
\href{http://sco.h-its.org/exelixis/resource/download/hands-on/Hands-On.tar.bz2}{here}
but the modified ones are in the
\href{https://github.com/PhyloMeth/LikelihoodTrees}{repository for this
PhyloMeth exercise} in the \texttt{/inst/extdata} folder.

Until now, we've seen NEXUS files, which can include data blocks. RAxML
uses Phylip-formatted files instead, which are simpler: a line that has
the number of taxa and the number of sites, followed by one line per
taxon with the taxon name, a space, and then the characters (though
there could be interleaving).

\section{Morphology search}\label{morphology-search}

First, we are going to examine morphology using likelihood. While
morphology is typically analyzed with parsimony, there are models for
morphology (i.e., Lewis 2001) and research suggests (Wright \& Hillis,
2014) that such models outperform parsimony for morphology, in addition
to being less prone (in theory) to long branch attraction (Felsenstein
1978). Therefore, absent strongly held concerns rooted in an
\href{http://onlinelibrary.wiley.com/doi/10.1111/cla.12148/full}{epistemological
paradigm}, it seems prudent to use a parametric model for morphology
(note this can be done in likelihood or Bayesian contexts).

Get the \href{https://github.com/PhyloMeth/LikelihoodTrees}{exercise}
and \textbf{complete the \texttt{InferMorphologyTree\_exercise} function
in \texttt{exercise.R}}. Also, look at the data in a text editor to get
a sense of the structure. Which taxa are going to be lumped into clades,
do you think? Some important things to note:

\begin{itemize}
\tightlist
\item
  Morphology (as well as some other data, such as SNPs) often includes
  only variable sites. This can cause a problem if not accounted for (it
  looks like all sites are evolving really fast, because the slow ones
  are ignored). There are corrections for this, three of which are
  implemented in RAxML.
\item
  RAxML creates a starting tree, then does a parsimony optimization,
  then likelihood. This is not a full parsimony search, though.
\item
  Remember that for nearly all tree searches, heuristic methods are
  used. That means that you are not guaranteed to get the best tree;
  given the size of tree space, one could almost say you're guaranteed
  not to find the best tree.
\item
  Computers are great at being logical. The downside is that they are
  terrible at being random. They often use the current time as a
  ``seed'' to get a pseudorandom number. You could think of it (this is
  more of an analogy than a description) as if the computer had a long
  table of stored ``random'' numbers, and that it started using numbers
  at the row corresponding to the number of seconds elapsed between the
  current time and some fixed date in the past. If you start two runs at
  different times, they'll have different numbers, but if you start them
  at the same time, they'll have the same ones. For tree search, there
  are often random moves: which branch is broken off and moved somewhere
  else. If you start two searches at the same time, thinking you're
  doing two independent searches, they'll perform exactly the same,
  despite the ``randomness''. RAxML asks users to supply a random number
  seed to it. If you use the same one across runs, they'll be exactly
  the same.
\end{itemize}

\section{DNA}\label{dna}

Most phylogenetic analyses for extant organisms use sequence data. This
is often presented as DNA, though sometimes the data are translated to
amino acids instead. Usually sequences from multiple genes are
concatenated. There are a wide range of models available for sequence
evolution. For DNA, the most popular remains general time reversible
(GTR): a model that allows for a different transition rate between every
pair of nucleotides, subject to the constraint that the rate from
nucleotide \emph{i} to \emph{j} is the same as the rate from \emph{j} to
\emph{i}. Different sites evolve at different rates (think of the sites
coding for the active site of an enzyme versus those in an intron that
has little to no functional purpose). One way to model this
heterogeneity is with a gamma distribution: the likelihood is evaluated
using several different rates for that site (Yang 1995). One can also
apply partitions: allow different sections of the data to have different
rates. This is commonly done to allow first, second, and third codon
positions to have different rates, or to allow different genes to have
different models of evolution. This can offer dramatic improvements in
the fit of a model to the data; it is especially important when dealing
with gappy data, such as cases where one gene is present for all taxa
but another gene has ben sampled for only a subset of taxa.

\textbf{Do InferDNATreeWithBootstrappingAndPartitions\_exercise() in the
homework}. Once this is done, install this homework library into R. From
the folder containing the homework:

\begin{verbatim}
R CMD INSTALL LikelihoodTrees
\end{verbatim}

Then, in R:

\begin{verbatim}
library(PhyloMethLikelihoodTrees)
results <- InferDNATreeWithBootstrappingAndPartitions_exercise()
\end{verbatim}

Though you may have to include other arguments (especially
\texttt{input.path}).

You can \textbf{plot your final tree}:

\begin{verbatim}
library(ape)
plot.phylo(results$ml.with.bs.tree, show.node.label=TRUE)
\end{verbatim}

This shows the branch lengths of the best ML tree and the bootstrap
proportions. This is from a non-parametric bootstrap (Felsenstein 1985):
the columns of data are sampled with replacement and then a tree search
is redone. The more times a bipartition (an edge) on a tree is
recovered, the more confidence we have in it (but this is not the same
as the probability of it being true). Note one common error: the numbers
reported, and in this case shown at nodes, are \emph{not} properties of
a node or a clade: they are bipartitions: taxa A, C, E fall attach
(perhaps through other nodes) to one end of an edge, and taxa B, D, E,
F, G, H are attached to the other end.

\chapter{Gene Tree Species Tree}\label{gene-tree-species-tree}

\section{Objectives}\label{objectives-8}

By the end of this chapter, you will:

\begin{itemize}
\tightlist
\item
  Understand why gene trees and species trees may not always agree
\item
  Know about some of the approaches in this area
\item
  Understand about phylogenetic networks
\end{itemize}

First, we will be looking at distinctions between gene trees and species
trees. For this, we'll be using Liang Liu's phybase package. This isn't
on CRAN, but it is available on his
\href{https://faculty.franklin.uga.edu/lliu/content/phybase?}{website}.
However, we'll be using a version I modified slightly and put on github.
Liang's package uses
\href{http://evolution.genetics.washington.edu/phylip/newicktree.html}{newick}
format (yes, named after the \href{http://newicks.com}{restaurant}), but
with additional options for including population size (branch width,
included as a \texttt{\#} sign followed by a number) as well as the more
traditional branch length. Both matter for coalescence: two copies are
much more likely to have coalesced on a long, narrow branch than a
short, fat one. Most packages in R instead use ape's \texttt{phylo}
format, so I wrote a few functions to deal with that. This version of
the package is on github; to install it:

\begin{verbatim}
library(devtools)
install_github("bomeara/phybase")
\end{verbatim}

For real science, Liang's is the canonical one (and definitely cite his)
but this will be easier for us to use for class.

First, get a tree from Open Tree of Life. We'll get a recent plant tree
from Beaulieu et al:

\begin{Shaded}
\begin{Highlighting}[]
\KeywordTok{library}\NormalTok{(rotl)}
\KeywordTok{library}\NormalTok{(ape)}
\NormalTok{phy <-}\StringTok{ }\KeywordTok{get_study_tree}\NormalTok{(}\StringTok{"ot_485"}\NormalTok{, }\StringTok{"tree1"}\NormalTok{)}
\KeywordTok{plot}\NormalTok{(phy, }\DataTypeTok{cex=}\FloatTok{0.3}\NormalTok{)}
\end{Highlighting}
\end{Shaded}

\includegraphics{comparative-methods_files/figure-latex/unnamed-chunk-17-1.pdf}

Let's simplify by dropping some taxa:

\begin{Shaded}
\begin{Highlighting}[]
\KeywordTok{library}\NormalTok{(geiger)}
\NormalTok{phy <-}\StringTok{ }\KeywordTok{drop.random}\NormalTok{(phy, }\KeywordTok{Ntip}\NormalTok{(phy) }\OperatorTok{-}\StringTok{ }\DecValTok{10}\NormalTok{)}
\KeywordTok{plot}\NormalTok{(phy)}
\KeywordTok{axisPhylo}\NormalTok{()}
\end{Highlighting}
\end{Shaded}

\includegraphics{comparative-methods_files/figure-latex/unnamed-chunk-18-1.pdf}

We can simulate gene trees on this tree:

\begin{Shaded}
\begin{Highlighting}[]
\KeywordTok{library}\NormalTok{(phybase)}
\NormalTok{gene.tree <-}\StringTok{ }\KeywordTok{sim.coaltree.phylo}\NormalTok{(phy, }\DataTypeTok{pop.size=}\FloatTok{1e-12}\NormalTok{)}
\KeywordTok{plot}\NormalTok{(gene.tree)}
\end{Highlighting}
\end{Shaded}

\includegraphics{comparative-methods_files/figure-latex/unnamed-chunk-19-1.pdf}

And it probably looks very similar to the initial tree:

\begin{Shaded}
\begin{Highlighting}[]
\KeywordTok{library}\NormalTok{(phytools)}
\KeywordTok{plot}\NormalTok{(}\KeywordTok{cophylo}\NormalTok{(phy, gene.tree, }\KeywordTok{cbind}\NormalTok{(}\KeywordTok{sort}\NormalTok{(phy}\OperatorTok{$}\NormalTok{tip.label), }\KeywordTok{sort}\NormalTok{(gene.tree}\OperatorTok{$}\NormalTok{tip.label))))}
\end{Highlighting}
\end{Shaded}

\begin{verbatim}
## Rotating nodes to optimize matching...
## Done.
\end{verbatim}

\includegraphics{comparative-methods_files/figure-latex/unnamed-chunk-20-1.pdf}

{[}Note I'm being a bit sloppy here: the initial branch lengths of the
tree we used are in millions of years (i.e., 5 = 5 MY) while the
coalescent sim is treating these as coalescent time units: there would
be even lower chance of incongruence if we converted the former into the
latter. Unfortunately, the simulator fails with branch lengths that are
realistically long for this tree.{]}

So, does this mean gene tree species tree issues aren't a problem?

Well, it depends on the details of the tree. One common misconception is
that gene tree species tree issues only relate to trees for recent
events. This problem can happen any time there are short, fat branches,
where lack of coalescence of copies can occur.

\begin{Shaded}
\begin{Highlighting}[]
\NormalTok{species.tree <-}\StringTok{ }\KeywordTok{rcoal}\NormalTok{(}\DecValTok{7}\NormalTok{)}
\NormalTok{species.tree}\OperatorTok{$}\NormalTok{edge.length <-}\StringTok{ }\NormalTok{species.tree}\OperatorTok{$}\NormalTok{edge.length }\OperatorTok{/}\StringTok{ }\NormalTok{(}\DecValTok{10}\OperatorTok{*}\KeywordTok{max}\NormalTok{(}\KeywordTok{branching.times}\NormalTok{(species.tree)))}
\NormalTok{gene.tree <-}\StringTok{ }\KeywordTok{sim.coaltree.phylo}\NormalTok{(species.tree)}
\KeywordTok{plot}\NormalTok{(}\KeywordTok{cophylo}\NormalTok{(species.tree, gene.tree, }\KeywordTok{cbind}\NormalTok{(}\KeywordTok{sort}\NormalTok{(species.tree}\OperatorTok{$}\NormalTok{tip.label), }\KeywordTok{sort}\NormalTok{(gene.tree}\OperatorTok{$}\NormalTok{tip.label))))}
\end{Highlighting}
\end{Shaded}

\begin{verbatim}
## Rotating nodes to optimize matching...
## Done.
\end{verbatim}

\includegraphics{comparative-methods_files/figure-latex/unnamed-chunk-21-1.pdf}

You should see (in most iterations), the above code giving a mismatch
between the gene tree and the species tree (the species tree has little
height). Now, let's lengthen the tips of the species tree:

\begin{Shaded}
\begin{Highlighting}[]
\NormalTok{tip.rows <-}\StringTok{ }\KeywordTok{which}\NormalTok{(species.tree}\OperatorTok{$}\NormalTok{edge[,}\DecValTok{2}\NormalTok{]}\OperatorTok{<=}\KeywordTok{Ntip}\NormalTok{(species.tree))}
\NormalTok{species.tree2 <-}\StringTok{ }\NormalTok{species.tree}
\NormalTok{species.tree2}\OperatorTok{$}\NormalTok{edge.length[tip.rows] <-}\StringTok{ }\DecValTok{100} \OperatorTok{+}\StringTok{ }\NormalTok{species.tree2}\OperatorTok{$}\NormalTok{edge.length[tip.rows]}
\NormalTok{gene.tree2 <-}\StringTok{ }\KeywordTok{sim.coaltree.phylo}\NormalTok{(species.tree2)}
\KeywordTok{plot}\NormalTok{(}\KeywordTok{cophylo}\NormalTok{(species.tree2, gene.tree2, }\KeywordTok{cbind}\NormalTok{(}\KeywordTok{sort}\NormalTok{(species.tree2}\OperatorTok{$}\NormalTok{tip.label), }\KeywordTok{sort}\NormalTok{(gene.tree2}\OperatorTok{$}\NormalTok{tip.label))))}
\end{Highlighting}
\end{Shaded}

\begin{verbatim}
## Rotating nodes to optimize matching...
## Done.
\end{verbatim}

\includegraphics{comparative-methods_files/figure-latex/unnamed-chunk-22-1.pdf}

It looks like a mismatch, but it's hard to see, since the tips are so
long. So plot the cladogram instead {[}we need to manually change branch
lengths to do this, though note we do not resimulate the gene tree{]}.

\begin{Shaded}
\begin{Highlighting}[]
\NormalTok{species.tree2.clado <-}\StringTok{ }\KeywordTok{compute.brlen}\NormalTok{(species.tree2)}
\NormalTok{gene.tree2.clado <-}\StringTok{ }\KeywordTok{compute.brlen}\NormalTok{(gene.tree2)}
\KeywordTok{plot}\NormalTok{(}\KeywordTok{cophylo}\NormalTok{(species.tree2.clado, gene.tree2.clado, }\KeywordTok{cbind}\NormalTok{(}\KeywordTok{sort}\NormalTok{(species.tree2.clado}\OperatorTok{$}\NormalTok{tip.label),}
\KeywordTok{sort}\NormalTok{(gene.tree2.clado}\OperatorTok{$}\NormalTok{tip.label))))}
\end{Highlighting}
\end{Shaded}

\begin{verbatim}
## Rotating nodes to optimize matching...
## Done.
\end{verbatim}

\includegraphics{comparative-methods_files/figure-latex/unnamed-chunk-23-1.pdf}

So we can see that even though the relevant divergences happened long
ago, gene tree species tree issues are still a problem.

\chapter{Dating}\label{dating}

\section{Objectives}\label{objectives-9}

By the end of this chapter, you will:

\begin{itemize}
\tightlist
\item
  Understand dating algorithms
\item
  Be able to use r8s and BEAST
\item
  Be afraid of calibrations
\end{itemize}

Make sure to \textbf{read the relevant papers}:
\url{https://www.mendeley.com/groups/8111971/phylometh/papers/added/0/tag/week5/}

To do this week's assignments, you will have to:

\begin{itemize}
\tightlist
\item
  \textbf{Download and install r8s} from
  \url{http://loco.biosci.arizona.edu/r8s/}
\item
  \textbf{Download BEAST2 and Beauti} (come together),
  \textbf{TreeAnnotator}, and \textbf{Tracer} from
  \url{http://beast2.org} and \url{http://beast.bio.ed.ac.uk/tracer}.
\item
  \textbf{Install a tweaked version of Geiger}

  \begin{itemize}
  \tightlist
  \item
    \texttt{library(devtools)}
  \item
    \texttt{install\_github("bomeara/geiger-v2")} (eventually I'll make
    a pull request)
  \end{itemize}
\end{itemize}

\section{BEAST}\label{beast}

For the BEAST part, we're not going to do our testing via R package
approach: dealing with file paths and such are too problematic. So we'll
just go through an exercise, but unlike many canned tutorials, you'll
have to figure out what to do at stages. Note that this is based on the
\href{https://github.com/CompEvol/beast2/blob/master/doc/tutorials/DivergenceDating/DivergenceDatingTutorialv2.0.3.pdf?raw=true}{tutorial}
by Drummond, Rambaut, and Bouckaert. A tutorial that provides far more
background info and other context than anything on the Beast2 website is
Tracy Heath's
\href{http://phyloworks.org/workshops/DivTime_BEAST2_tutorial_FBD.pdf}{tutorial}
from the Bodega Bay Workshop in Applied Phylogenetics: if you are going
to run BEAST, read that. There is also now a
\href{http://beast2.org/book/}{book} that you can buy.

BEAST uses XML files for commands, rather than NEXUS files (though it
does use NEXUS files for data). This XML format is different from
\href{http://www.nexml.org}{NeXML} and
\href{http://www.phyloxml.org}{PhyloXML}, two other XML formats proposed
for phylogenetics (though, as of now, they're still not used much in the
field -- most students won't encounter them). These files are often made
using the program BEAUTi (also from the BEAST developers) and then,
sometimes, hand edited.

\begin{enumerate}
\def\labelenumi{\arabic{enumi}.}
\tightlist
\item
  \textbf{Import primates-mtDNA.nex} into BEAUTi; it's includided in
  examples/nexus in the BEAST folder. Note it's File -\textgreater{}
  Import alignment.
\item
  But wait -- what are you loading? Have you looked at the file? Open it
  in a text editor (or R, or Mesquite) and look at it. Do you believe
  the sequences are aligned properly? Is there anything weird about
  them? This is an essential step.
\item
  We will be partitioning, as with RAxML. This file already has some
  partitions, but some overlap (coding vs the three codon positions).
  Delete the coding partition using the \texttt{-} sign button on the
  bottom of the window.
\item
  Partitions can be linked: allow different rates but the same topology,
  for example. Select all four partitions and click Link Trees. Then
  click on the first pull down (for noncoding) for the Tree column and
  name it ``tree''. Do the same for Clock (rename to ``clock'').
\item
  Temporarily link sites in the same way.
\item
  BEAST has a variety of models. We're going to do HKY (so, two
  different rates) with gamma-distributed rate differences. To do this:

  \begin{itemize}
  \tightlist
  \item
    Set Gamma Category Count to 4
  \item
    Set the Shape to be estimated
  \item
    Select the HKY model
  \item
    Make sure Kappa is estimated
  \item
    Make sure frequences are estimated
  \item
    Select estimate for Substitution Rate
  \end{itemize}
\item
  Go back to partitions and click on Unlink Site Models. This lets each
  partition have its own HKY+gamma model (and doing the link
  -\textgreater{} set -\textgreater{} unlink lets you save on work of
  setting it for each one).
\item
  We need to choose a clock model. A Strict Clock is the classic
  molecular clock. Most people using BEAST (and this is based on various
  sim studies) use relaxed clock log normal, so choose that. Having the
  number of discrete rates set to -1 will allow as many rates as
  branches.
\item
  Now comes the tricky bit: setting your priors. For example, you need a
  prior for the tree: assume a Yule prior (only speciation events, no
  extinction)? Or a birth-death one? {[}and think about the implications
  of these choices for later analyses -- estimating extinction rates,
  for example{]}. And even if you have a belief about whether extinction
  might have happened or not, what about parameters like gamma shape for
  first codon position sites? Exponential, beta, etc.? You probably
  don't have a good idea of what they should be in terms of shape, let
  alone priors for the actual values. And this might affect your
  analyses: the result is due to the prior and the likelihood. Let's
  leave all the priors set to the default for now, except for the tree:
  do a birth-death model for that.
\item
  We can also add priors: say, a prior for the age of a node.

  \begin{itemize}
  \tightlist
  \item
    Click on the ``+'' button
  \item
    Call the Taxon set label ``human-chimp''
  \item
    Click on Homo\_sapiens and then the ``\textgreater{}\textgreater{}''
    button
  \item
    Do the same for Pan
  \item
    Click ok
  \item
    Force it to be monophyletic
  \item
    Choose Log Normal for the age prior
  \item
    Select mean in real space (so it's easier for us to understand the
    age)
  \item
    Enter 6 (for 6 MY) for the M = mean of the distribution
  \item
    \textbf{Select a value for S that leads to a 95\% range of about 5-7
    MY (use the information on the right side of the window to help)}
  \end{itemize}
\item
  Go to MCMC to set parameters for the run

  \begin{itemize}
  \tightlist
  \item
    Do 1M generations to start
  \item
    You can also set how often the info is saved to disk or printed on
    the screen. Change the tracelog to primates\_birthdeath\_prior.log
  \end{itemize}
\item
  Save this in the same folder as your original nexus file: maybe store
  as primates\_birthdeath\_prior.xml.
\item
  Yay! We have created a file with commands to run BEAST. Open it in a
  text editor and look at it (don't modify or save it).
\item
  Now open BEAST. Choose the xml file you just made and run.
\item
  Time passes.
\item
  Now open the log file in Tracer. Investigate some of the statistics,
  including looking at ESS: effective sample size. Ones that aren't
  black indicate ones that did not run long enough.
\item
  Go back to BEAUTi.

  \begin{itemize}
  \tightlist
  \item
    Change the tree prior to a Yule tree
  \item
    Change the tracelog to primates\_yule\_prior.log and change the tree
    file name
  \item
    Save as the file to primates\_yule\_prior.xml
  \end{itemize}
\item
  Use this new xml file to run BEAST again.
\item
  Now look at this log file with the other one, both in Tracer. Are the
  estimates the same? Even for something like tree height? What does
  this suggest?
\item
  Use TreeAnnotator on one of your tree files to summarize.

  \begin{itemize}
  \tightlist
  \item
    Decide what the burnin should be: that is the number of trees (not
    number of generations) to delete.
  \item
    Change Posterior probability limit to 0: if you are going to show an
    edge, you should show the support for that. Many people only show
    support above 50\% but still show all branches: this is problematic
    (only showing uncertainty on the edges where uncertainty is
    relatively low).
  \end{itemize}
\item
  FigTree or R can be used to visualize the final tree with support.
\end{enumerate}

\section{r8s}\label{r8s}

r8s implements several functions for converting a phylogram to a
chronogram. It does the classic, Langley-Fitch molecular clock which
stretches branches but assumes a constant rate for all. It also
implements two algorithms by Sanderson: nonparametric rate smoothing
(NPRS) and penalized likelihood (PL). Both relax the assumption of
constant rate of evolution and instead allow rates to vary along the
tree. NPRS tries to minimize rate changes at nodes. PL has a model for
changes (to give likelihood of original branch lengths) and combines
this with a nonparametric penalty for rate changes. It tries to minimize
the combination of these two parameters, but there is a user-set penalty
to decide the relative value of these in the combined sum. This is set
by cross-validation: delete some data, estimate parameters, predict the
deleted data, and see how close the deleted data are to the simulated
data. In this case, the datum deleted is a single tip, and the length of
this branch is the value to predict. This can now happen within r8s. In
general, PL is more accurate than NPRS, but is slower (but both are much
faster than BEAST). \texttt{treepl} is a later program that implements
Sanderson's algorithms but can work on much larger trees.

Jon Eastman coded an interface to this in Geiger, but it wasn't exposed
to users. I've added some additional features (including an
\texttt{ez.run} mode) and documentation to his code. This is now in a
fork of Geiger, but I'll file a pull request soon to put it into the
main code. For now, make sure you have installed the forked version:

\begin{verbatim}
devtools::install_github("bomeara/geiger-v2")
library(geiger)
\end{verbatim}

Then use the help for \texttt{?r8s.phylo} to figure out how to use this
function. Also look at the
\href{http://loco.biosci.arizona.edu/r8s/r8s1.7.manual.pdf}{r8s manual}
to understand the options.

\textbf{Run the examples} in Geiger for this. You can also look at the
examples that come with r8s.

\section{Applying to your own work}\label{applying-to-your-own-work}

By this point in the course, you should be thirsting to apply these
tools to your own questions. Do so! \textbf{Get a dataset (think back to
the getting trees method), infer a tree (if needed), and date it using
one of these approaches.} I'd advise making your own github repository
for this. You could pay to keep it secret; I'd advise it's probably not
worth it (there is some risk of being scooped, but it's pretty low) but
it's your call.

\chapter{Visualizing trees and trees with
data}\label{visualizing-trees-and-trees-with-data}

\section{Objectives}\label{objectives-10}

By the end of this chapter, you will:

\begin{itemize}
\tightlist
\item
  Understand ways to visualize trees
\item
  Understand how to visualize trees with data
\end{itemize}

It's always important to visualize trees, and data on your trees. For
example, most comparative methods require branch lengths. Are yours
reasonable? Do you have any taxa on very long branches (which could
indicate alignment or paralogy issues)? Are there many effectively zero
length branches? Does everything agree with what you know of taxonomy?

To start, let's take a sample tree: a tree of snakes by Pyron R.A.,
Burbrink F., Colli G., Montes de oca A.N., Vitt L.J., Kuczynski C.A., \&
Wiens J.J. 2010. The phylogeny of advanced snakes (Colubroidea), with
discovery of a new subfamily and comparison of support methods for
likelihood trees. Molecular Phylogenetics and Evolution 58 (2): 329-342.
{[}\#TODO: add proper citation{]}. It is a tree of 767 taxa. But we'll
start with a 12 tip subtree.

The natural way you'd plot this in R:

\begin{Shaded}
\begin{Highlighting}[]
\NormalTok{ape}\OperatorTok{::}\KeywordTok{plot.phylo}\NormalTok{(small.phy)}
\end{Highlighting}
\end{Shaded}

\includegraphics{comparative-methods_files/figure-latex/unnamed-chunk-26-1.pdf}

Plotting it with tips to the right is most common, but there are other
options, too:

\begin{Shaded}
\begin{Highlighting}[]
\NormalTok{ape}\OperatorTok{::}\KeywordTok{plot.phylo}\NormalTok{(small.phy, }\DataTypeTok{direction=}\StringTok{"upwards"}\NormalTok{)}
\end{Highlighting}
\end{Shaded}

\includegraphics{comparative-methods_files/figure-latex/unnamed-chunk-27-1.pdf}

Especially for big trees, fan (circle trees) can also be popular:

\begin{Shaded}
\begin{Highlighting}[]
\NormalTok{ape}\OperatorTok{::}\KeywordTok{plot.phylo}\NormalTok{(small.phy, }\DataTypeTok{type=}\StringTok{"fan"}\NormalTok{)}
\end{Highlighting}
\end{Shaded}

\includegraphics{comparative-methods_files/figure-latex/unnamed-chunk-28-1.pdf}
Sometimes for just seeing the tree structure itself, once can remove
branch lengths:

\begin{Shaded}
\begin{Highlighting}[]
\NormalTok{ape}\OperatorTok{::}\KeywordTok{plot.phylo}\NormalTok{(small.phy, }\DataTypeTok{type=}\StringTok{"cladogram"}\NormalTok{)}
\end{Highlighting}
\end{Shaded}

\includegraphics{comparative-methods_files/figure-latex/unnamed-chunk-29-1.pdf}

\chapter{Appendix}\label{appendix}

Maybe include vignettes: see
\url{http://stackoverflow.com/questions/17593912/insert-portions-of-a-markdown-document-inside-another-markdown-document-using-kn}

\bibliography{packages.bib,book.bib}


\end{document}
